% Options for packages loaded elsewhere
\PassOptionsToPackage{unicode}{hyperref}
\PassOptionsToPackage{hyphens}{url}
%
\documentclass[
]{article}
\usepackage{amsmath,amssymb}
\usepackage{iftex}
\ifPDFTeX
  \usepackage[T1]{fontenc}
  \usepackage[utf8]{inputenc}
  \usepackage{textcomp} % provide euro and other symbols
\else % if luatex or xetex
  \usepackage{unicode-math} % this also loads fontspec
  \defaultfontfeatures{Scale=MatchLowercase}
  \defaultfontfeatures[\rmfamily]{Ligatures=TeX,Scale=1}
\fi
\usepackage{lmodern}
\ifPDFTeX\else
  % xetex/luatex font selection
\fi
% Use upquote if available, for straight quotes in verbatim environments
\IfFileExists{upquote.sty}{\usepackage{upquote}}{}
\IfFileExists{microtype.sty}{% use microtype if available
  \usepackage[]{microtype}
  \UseMicrotypeSet[protrusion]{basicmath} % disable protrusion for tt fonts
}{}
\makeatletter
\@ifundefined{KOMAClassName}{% if non-KOMA class
  \IfFileExists{parskip.sty}{%
    \usepackage{parskip}
  }{% else
    \setlength{\parindent}{0pt}
    \setlength{\parskip}{6pt plus 2pt minus 1pt}}
}{% if KOMA class
  \KOMAoptions{parskip=half}}
\makeatother
\usepackage{xcolor}
\usepackage[margin=1in]{geometry}
\usepackage{color}
\usepackage{fancyvrb}
\newcommand{\VerbBar}{|}
\newcommand{\VERB}{\Verb[commandchars=\\\{\}]}
\DefineVerbatimEnvironment{Highlighting}{Verbatim}{commandchars=\\\{\}}
% Add ',fontsize=\small' for more characters per line
\usepackage{framed}
\definecolor{shadecolor}{RGB}{248,248,248}
\newenvironment{Shaded}{\begin{snugshade}}{\end{snugshade}}
\newcommand{\AlertTok}[1]{\textcolor[rgb]{0.94,0.16,0.16}{#1}}
\newcommand{\AnnotationTok}[1]{\textcolor[rgb]{0.56,0.35,0.01}{\textbf{\textit{#1}}}}
\newcommand{\AttributeTok}[1]{\textcolor[rgb]{0.13,0.29,0.53}{#1}}
\newcommand{\BaseNTok}[1]{\textcolor[rgb]{0.00,0.00,0.81}{#1}}
\newcommand{\BuiltInTok}[1]{#1}
\newcommand{\CharTok}[1]{\textcolor[rgb]{0.31,0.60,0.02}{#1}}
\newcommand{\CommentTok}[1]{\textcolor[rgb]{0.56,0.35,0.01}{\textit{#1}}}
\newcommand{\CommentVarTok}[1]{\textcolor[rgb]{0.56,0.35,0.01}{\textbf{\textit{#1}}}}
\newcommand{\ConstantTok}[1]{\textcolor[rgb]{0.56,0.35,0.01}{#1}}
\newcommand{\ControlFlowTok}[1]{\textcolor[rgb]{0.13,0.29,0.53}{\textbf{#1}}}
\newcommand{\DataTypeTok}[1]{\textcolor[rgb]{0.13,0.29,0.53}{#1}}
\newcommand{\DecValTok}[1]{\textcolor[rgb]{0.00,0.00,0.81}{#1}}
\newcommand{\DocumentationTok}[1]{\textcolor[rgb]{0.56,0.35,0.01}{\textbf{\textit{#1}}}}
\newcommand{\ErrorTok}[1]{\textcolor[rgb]{0.64,0.00,0.00}{\textbf{#1}}}
\newcommand{\ExtensionTok}[1]{#1}
\newcommand{\FloatTok}[1]{\textcolor[rgb]{0.00,0.00,0.81}{#1}}
\newcommand{\FunctionTok}[1]{\textcolor[rgb]{0.13,0.29,0.53}{\textbf{#1}}}
\newcommand{\ImportTok}[1]{#1}
\newcommand{\InformationTok}[1]{\textcolor[rgb]{0.56,0.35,0.01}{\textbf{\textit{#1}}}}
\newcommand{\KeywordTok}[1]{\textcolor[rgb]{0.13,0.29,0.53}{\textbf{#1}}}
\newcommand{\NormalTok}[1]{#1}
\newcommand{\OperatorTok}[1]{\textcolor[rgb]{0.81,0.36,0.00}{\textbf{#1}}}
\newcommand{\OtherTok}[1]{\textcolor[rgb]{0.56,0.35,0.01}{#1}}
\newcommand{\PreprocessorTok}[1]{\textcolor[rgb]{0.56,0.35,0.01}{\textit{#1}}}
\newcommand{\RegionMarkerTok}[1]{#1}
\newcommand{\SpecialCharTok}[1]{\textcolor[rgb]{0.81,0.36,0.00}{\textbf{#1}}}
\newcommand{\SpecialStringTok}[1]{\textcolor[rgb]{0.31,0.60,0.02}{#1}}
\newcommand{\StringTok}[1]{\textcolor[rgb]{0.31,0.60,0.02}{#1}}
\newcommand{\VariableTok}[1]{\textcolor[rgb]{0.00,0.00,0.00}{#1}}
\newcommand{\VerbatimStringTok}[1]{\textcolor[rgb]{0.31,0.60,0.02}{#1}}
\newcommand{\WarningTok}[1]{\textcolor[rgb]{0.56,0.35,0.01}{\textbf{\textit{#1}}}}
\usepackage{graphicx}
\makeatletter
\def\maxwidth{\ifdim\Gin@nat@width>\linewidth\linewidth\else\Gin@nat@width\fi}
\def\maxheight{\ifdim\Gin@nat@height>\textheight\textheight\else\Gin@nat@height\fi}
\makeatother
% Scale images if necessary, so that they will not overflow the page
% margins by default, and it is still possible to overwrite the defaults
% using explicit options in \includegraphics[width, height, ...]{}
\setkeys{Gin}{width=\maxwidth,height=\maxheight,keepaspectratio}
% Set default figure placement to htbp
\makeatletter
\def\fps@figure{htbp}
\makeatother
\setlength{\emergencystretch}{3em} % prevent overfull lines
\providecommand{\tightlist}{%
  \setlength{\itemsep}{0pt}\setlength{\parskip}{0pt}}
\setcounter{secnumdepth}{-\maxdimen} % remove section numbering
\ifLuaTeX
  \usepackage{selnolig}  % disable illegal ligatures
\fi
\usepackage{bookmark}
\IfFileExists{xurl.sty}{\usepackage{xurl}}{} % add URL line breaks if available
\urlstyle{same}
\hypersetup{
  pdftitle={Assessing whether there was an association between anxiety levels and threat estimation during the Covid-19 pandemic across four European countries},
  hidelinks,
  pdfcreator={LaTeX via pandoc}}

\title{Assessing whether there was an association between anxiety levels
and threat estimation during the Covid-19 pandemic across four European
countries}
\author{}
\date{\vspace{-2.5em}2024-11-26}

\begin{document}
\maketitle

{
\setcounter{tocdepth}{2}
\tableofcontents
}
\includegraphics{markdown_image/corona-5401250_1280.jpg}

\begin{Shaded}
\begin{Highlighting}[]
\NormalTok{knitr}\SpecialCharTok{::}\NormalTok{opts\_chunk}\SpecialCharTok{$}\FunctionTok{set}\NormalTok{(}\AttributeTok{echo =} \ConstantTok{TRUE}\NormalTok{)}

\FunctionTok{library}\NormalTok{(tidyverse)}
\end{Highlighting}
\end{Shaded}

\begin{verbatim}
## Warning: package 'lubridate' was built under R version 4.4.2
\end{verbatim}

\begin{verbatim}
## -- Attaching core tidyverse packages ------------------------ tidyverse 2.0.0 --
## v dplyr     1.1.4     v readr     2.1.5
## v forcats   1.0.0     v stringr   1.5.1
## v ggplot2   3.5.1     v tibble    3.2.1
## v lubridate 1.9.4     v tidyr     1.3.1
## v purrr     1.0.2     
## -- Conflicts ------------------------------------------ tidyverse_conflicts() --
## x dplyr::filter() masks stats::filter()
## x dplyr::lag()    masks stats::lag()
## i Use the conflicted package (<http://conflicted.r-lib.org/>) to force all conflicts to become errors
\end{verbatim}

\begin{Shaded}
\begin{Highlighting}[]
\FunctionTok{library}\NormalTok{(here)}
\end{Highlighting}
\end{Shaded}

\begin{verbatim}
## here() starts at C:/Users/kiren/OneDrive/Documents/psy6422_assessment/covid
\end{verbatim}

\begin{Shaded}
\begin{Highlighting}[]
\FunctionTok{library}\NormalTok{(dplyr)}
\FunctionTok{library}\NormalTok{(ggplot2)}
\FunctionTok{library}\NormalTok{(extrafont)}
\end{Highlighting}
\end{Shaded}

\begin{verbatim}
## Registering fonts with R
\end{verbatim}

\subsection{Project Description}\label{project-description}

This project is an assessed piece of work as part of the MSc
Psychological Research Methods with Data Science course at the
University of Sheffield. The assessment requires students to produce a
visualisation of an open source dataset using R. In line with the
movement towards increasing the reproducibility of scientific findings,
the processes employed for the of data visualisation, along with the
associated code for the exploratory data analysis are reported.

\subsubsection{Background}\label{background}

The entire world was impacted by the effects of the Covid-19
(SARS-CoV-2) virus as it resulted in a pandemic. Following the rapid
spread of the disease, many countries took action by implementing
lockdown measures and introducing social distancing and hygiene
protocols (Miyah et al., 2022).

The data collected during this period is plentiful and has provided new
insights into human behaviour when faced with the uncertainty of a novel
and harmful disease. Using an online survey, Abadi et al., (2023)
obtained data from 2031 participants from the UK, Spain, Germany and
Netherlands. The authors report that they chose to include these
specific countries on the basis that each differed significantly on the
number of Covid-19 related deaths at the time of the survey completion,
where Spain had the most number of cases, followed by the UK,the
Netherlands and Germany. Moreover, the implementation of lockdown by
these countries' respective governments and the socio-political
attitudes within their differed. Participants who engaged with the
survey completed measures on several variables of interest, such as
socio-political attitudes, conspiracy mentality and perceptions of
threat, to name a few.

\subsection{Data Origins}\label{data-origins}

The original article was retrieved from the Journal of Open Psychology
Data. Click
\href{https://openpsychologydata.metajnl.com/articles/10.5334/jopd.86\#4}{here}
to view the full article.

The data and codebook was retrieved from the authors'\emph{figshare}
repository which is available
\href{https://uvaauas.figshare.com/articles/dataset/A_Dataset_of_Social-Psychological_and_Emotional_Reactions_during_the_COVID-19_Pandemic_across_Four_European_Countries/17085719?file=41458140}{here}.

The authors of the paper have indicated that their dataset is published
under the CC BY-SA license. This allows the re-use of the data, provided
that the original authors are acknowledged.

The URL is this visualisation project is
\href{https://ijames8.github.io/covid/}{here}

The repository for these pages is:
\url{https://github.com/IJames8/covid.git}

\subsubsection{Import the data}\label{import-the-data}

\begin{Shaded}
\begin{Highlighting}[]
\NormalTok{raw\_data }\OtherTok{\textless{}{-}} \FunctionTok{read.csv}\NormalTok{(}\StringTok{"data/raw\_data.csv"}\NormalTok{)}

\FunctionTok{head}\NormalTok{(raw\_data)}
\end{Highlighting}
\end{Shaded}

\begin{verbatim}
##   RespondentID Country A3.1 A3.2 A3.3 A3.4 A3.5 A3.6 A3.7 A3.8 A3.9 A3.10 A5.1
## 1            1       1    2    3   52   52    4    1    5   12    1     4    7
## 2            2       1    2    5   52   52    3    1    8    2    5     2   10
## 3            3       1    1    4   94   94    4    1    5    2    1     4   10
## 4            4       1    1    2  221  221    1    1   10    3    4     5    8
## 5            5       1    1    3   52   52    3    1    8    2    0     6    7
## 6            6       1    1    1   52   52    1    4    9    3    4     2    8
##   A5.2 A5.3. A5.4 A5.5 A6.1 A7.1 A7.2 A9.1 A9.2 A9.3. A9.4 A9.5 A9.6 A9.7 A9.8.
## 1    3     1    3    8    7    2    2    2    1     1    3    3    2    5     1
## 2   10     1    8    1    5    2    2    4    3     4    6    1    1    1     6
## 3   10     1   10    7    6    2    2    5    4     3    7    4    7    7     7
## 4    8     9    7    7    8    4    4    7    6     5    6    7    7    6     6
## 5    7     2    6    3    7    1    1    1    1     5    3    5    1    1     6
## 6   10     9    9    9    4    3    4    6    6     7    7    6    5    7     5
##   A11.1 A11.2 A11.3 A11.4 A12.1 A12.2 A12.3 A12.4 A13.1 A13.2 A13.3 A13.4 A14.1
## 1     2     2     2     2     1     1     1     2     2     2     2     2     2
## 2     1     1     1     1     1     1     1     1     1     1     1     1     1
## 3     1     1     1     1     1     1     1     1     1     1     1     2     1
## 4     2     1     1     2     2     1     1     2     1     1     2     2     2
## 5     1     2     1     1     1     1     1     1     1     2     1     2     2
## 6     1     2     1     1     1     2     1     2     2     1     1     2     1
##   A14.2 A14.3 A14.4 A16.1 A16.2 A16.3. A16.4. A16.5 A16.6 A17.1 A17.2 A17.3
## 1     1     2     2     5     2      9      9     7     7     5     2     2
## 2     1     2     1     4     8      2      2    10     9     9     9     8
## 3     1     1     2     7     7      7      3    10    10     8    10     6
## 4     2     2     2     7     8      2      2     8     8     8     8     9
## 5     2     2     2     6     2      4      4     7     5     4     3     8
## 6     2     1     2     9    10      4      7     5     7     9     6     9
##   A17.4 A17.5 A17.6 A17.7 A17.8 A17.9 A17.10. A17.11 A17.12 A18.1 A20.1 A20.2
## 1     5     3     2     6    10     8       9      7      7     5     7     7
## 2     9     6     5     1     1     1       2      9      8     5     7     7
## 3     8    10    10    10     7    10       4     10      9     5     7     5
## 4     8     7     9     7     8     7       2      8      8     5     7     7
## 5     6     4     3     7     2     2       3      5      3     5     6     6
## 6    10     8    10     8    10     9       4      6      9     5     6     7
##   A20.3 A21.1 A21.2. A22.1 A22.2 A23.1 A23.2 A23.3 A24.1 A24.2 A24.3 A24.4
## 1     5     4      2     3     7     6     6     6     5     5     5     4
## 2     7     7      7     7     7     7     7     7     7     7     7     7
## 3     7     6      3     5     6     7     7     5     7     7     7     4
## 4     6     7      6     6     7     7     6     7     6     7     6     7
## 5     5     5      5     6     5     5     5     6     7     7     5     5
## 6     7     6      7     7     7     7     7     5     7     7     5     6
##   A24.5 A24.6 A25.1 A25.2 A27.1 A27.2. A27.3 A27.4 A27.5 A27.6. A27.7. A27.8
## 1     4     1     1     1     4      5     2     4     2      5      5     7
## 2     7     7     8     8     2      6     2     2     1      7      1     7
## 3     2     7    10     6     7      7     7     6     5      5      7     7
## 4     6     5     7     8     7      6     7     7     6      6      6     5
## 5     6     5     3     2     6      6     7     6     6      6      6     4
## 6     7     5    10     8     5      7     7     7     5      7      6     6
##   A27.9 A27.10 A27.11 A29.1 A29.2 A29.3 A29.4 A29.5 A29.6 A29.7 A29.8 A29.9
## 1     7      5      5     7     7     7     7     7     7     7     7     7
## 2     7      2      6     7     7     7     7     7     7     7     7     7
## 3     5      7      5     7     7     7     7     7     7     7     7     7
## 4     7      7      6     5     7     6     6     7     7     6     7     6
## 5     7      6      5     5     5     6     6     6     6     6     6     7
## 6     6      6      7     7     6     7     5     7     5     6     6     7
##   A31.1 A31.2 A31.3 A31.4 A31.5 A31.6 A31.7 A31.8 A31.9 A32.1 A32.2 A34.1 A34.2
## 1     7     7     7     1     1     7     4     1     1     1     1     1     2
## 2     7     7     7     7     1     7     7     1     1     1    10     6     6
## 3     7     7     7     6     7     7     7     7     7    10     5     2     6
## 4     6     7     7     6     5     7     6     6     4     8     9     7     7
## 5     7     6     7     1     7     6     7     7     6     3     8     2     3
## 6     7     7     5     4     4     7     6     5     6     9    10     7     5
##   A34.3 A34.4 A36.1 A36.2 A36.3 A36.4. A36.5 A36.6 A38.1. A38.2. A38.3 A38.4
## 1     2     4     5     7     7      1     6     6      4      1     7     1
## 2     4     6     1     1     1      5     4     1      1      7     7     1
## 3     4     1     3     2     6      6     1     1      5      2     6     2
## 4     7     6     6     5     7      7     7     6      7      6     6     7
## 5     2     2     6     5     6      2     5     6      5      3     4     1
## 6     7     7     7     4     5      7     6     6      6      6     7     5
##   A38.5 A40.1 A40.2 A40.3 A40.4 A40.5 A40.6 A40.7 A40.8 A40.9 A40.10 A40.11
## 1     2     4     7     5     7     5     7     7     6     1      3      7
## 2     1     7     7     6     7     7     7     7     7     7      1      7
## 3     6     7     7     7     7     7     7     7     7     7      7      7
## 4     7     5     6     6     4     5     7     6     7     7      6      7
## 5     5     6     6     6     7     5     5     2     6     5      5      7
## 6     7     6     6     5     7     5     7     5     7     6      6      7
\end{verbatim}

\begin{Shaded}
\begin{Highlighting}[]
\CommentTok{\# check that the data has imported properly. The number of observations and variables appears to be correct. }
\end{Highlighting}
\end{Shaded}

\subsection{Research Question}\label{research-question}

In reviewing the reuse potential of the data as suggested by the Abadi
et al., (2013), this data visualisation project aims to investigate
whether anxiety around the coronavirus, as indicated by the scores on
the anxiety measure, is associated with an overestimation of various
threats, such as those related to coronavirus, climate,
symbolic/material or safety, as indicated by the scores on the threat
estimation measure for each participant.

\subsection{Data Preparation}\label{data-preparation}

For this particular visualisation, the codebook indicates that the
variables of interest are items 5.1 -5.5. These relate to anxiety about
coronavirus. It is important to note that item 5.3. is reverse coded.
Variables 16.1 - 16.3 (threat estimation around cornoavirus) and 17.1 -
17.12 (threat estimation around climate, symbolic, material and safety)
are also needed as these relate to the scores around threat estimation.
Within this category, items 16.3, 16.4 and 17.10 are reverse coded.

The remaining items will be removed from the dataframe, leaving only
information about the relevant variables, along with annonymised
participant ID and country.

\subsubsection{Wrangle the data}\label{wrangle-the-data}

\begin{Shaded}
\begin{Highlighting}[]
\CommentTok{\# remove the unwanted variables. }

\NormalTok{df1 }\OtherTok{\textless{}{-}}\NormalTok{ raw\_data }\SpecialCharTok{\%\textgreater{}\%} 
  \FunctionTok{select}\NormalTok{(}\SpecialCharTok{{-}}\NormalTok{A3}\FloatTok{.1} \SpecialCharTok{:} \SpecialCharTok{{-}}\NormalTok{A3}\FloatTok{.10}\NormalTok{, }\SpecialCharTok{{-}}\NormalTok{A6}\FloatTok{.1} \SpecialCharTok{:} \SpecialCharTok{{-}}\NormalTok{A14}\FloatTok{.4}\NormalTok{, }\SpecialCharTok{{-}}\NormalTok{A18}\FloatTok{.1}\SpecialCharTok{:{-}}\NormalTok{A40}\FloatTok{.11}\NormalTok{)}
\end{Highlighting}
\end{Shaded}

\begin{Shaded}
\begin{Highlighting}[]
\CommentTok{\# Items 5.3, 16.3, 16.4 and 17.10, are reverse coded. This will need to be adjusted in the dataframe.}
\CommentTok{\# Item 5.3. is scored on a scale from 1 {-} 10. High scores indicate, high levels of anxiety.}
\CommentTok{\# Item 16.4 and 16.3 are scored on a scale from 1 {-} 10. High scores indicate, high levels of threat estimation coronavirus.}
\CommentTok{\# Item 17.10. is scored on a scale from 1 {-} 10. High scores indicate, high levels of threat estimation. }

\CommentTok{\#Create function to reverse score the items indicated in the codebook. Utilising the " N{-}plus{-}one{-}minus{-}x" method, the new score is calculated by subtracting the obtained raw score (x) from 11 since the scale is from 1 to 10 (n{-}plus{-}1). }

\NormalTok{reversed\_scores }\OtherTok{\textless{}{-}} \ControlFlowTok{function}\NormalTok{(df1, columns, }\AttributeTok{max\_score =} \DecValTok{11}\NormalTok{) \{}
\NormalTok{  df1 }\SpecialCharTok{\%\textgreater{}\%} 
    \FunctionTok{mutate}\NormalTok{(}\FunctionTok{across}\NormalTok{(}\FunctionTok{all\_of}\NormalTok{(columns), }\SpecialCharTok{\textasciitilde{}}\NormalTok{ max\_score }\SpecialCharTok{{-}}\NormalTok{ ., }\AttributeTok{.names =} \StringTok{"\{.col\}\_reversed"}\NormalTok{))}
\NormalTok{  \}}

\CommentTok{\# Apply the function to relevant items and assign this to an updated dataframe.}
\NormalTok{df2 }\OtherTok{\textless{}{-}} \FunctionTok{reversed\_scores}\NormalTok{(df1, }\FunctionTok{c}\NormalTok{(}\StringTok{"A5.3."}\NormalTok{, }\StringTok{"A16.3."}\NormalTok{, }\StringTok{"A16.4."}\NormalTok{, }\StringTok{"A17.10."}\NormalTok{))}


\CommentTok{\# View the dataframe to check that new columns have been added to df2 for the reversed scored items. Check that the reversed scores look right and makes sense. }

\FunctionTok{head}\NormalTok{(df2)}
\end{Highlighting}
\end{Shaded}

\begin{verbatim}
##   RespondentID Country A5.1 A5.2 A5.3. A5.4 A5.5 A16.1 A16.2 A16.3. A16.4.
## 1            1       1    7    3     1    3    8     5     2      9      9
## 2            2       1   10   10     1    8    1     4     8      2      2
## 3            3       1   10   10     1   10    7     7     7      7      3
## 4            4       1    8    8     9    7    7     7     8      2      2
## 5            5       1    7    7     2    6    3     6     2      4      4
## 6            6       1    8   10     9    9    9     9    10      4      7
##   A16.5 A16.6 A17.1 A17.2 A17.3 A17.4 A17.5 A17.6 A17.7 A17.8 A17.9 A17.10.
## 1     7     7     5     2     2     5     3     2     6    10     8       9
## 2    10     9     9     9     8     9     6     5     1     1     1       2
## 3    10    10     8    10     6     8    10    10    10     7    10       4
## 4     8     8     8     8     9     8     7     9     7     8     7       2
## 5     7     5     4     3     8     6     4     3     7     2     2       3
## 6     5     7     9     6     9    10     8    10     8    10     9       4
##   A17.11 A17.12 A5.3._reversed A16.3._reversed A16.4._reversed A17.10._reversed
## 1      7      7             10               2               2                2
## 2      9      8             10               9               9                9
## 3     10      9             10               4               8                7
## 4      8      8              2               9               9                9
## 5      5      3              9               7               7                8
## 6      6      9              2               7               4                7
\end{verbatim}

\begin{Shaded}
\begin{Highlighting}[]
\CommentTok{\# Remove old scores and replace with reversed scores.}

\NormalTok{df2}\SpecialCharTok{$}\NormalTok{A5.}\FloatTok{3.} \OtherTok{\textless{}{-}}\NormalTok{ df2}\SpecialCharTok{$}\NormalTok{A5.}\FloatTok{3.}\NormalTok{\_reversed  }
\NormalTok{df2}\SpecialCharTok{$}\NormalTok{A16.}\FloatTok{3.} \OtherTok{\textless{}{-}}\NormalTok{ df2}\SpecialCharTok{$}\NormalTok{A16.}\FloatTok{3.}\NormalTok{\_reversed}
\NormalTok{df2}\SpecialCharTok{$}\NormalTok{A16.}\FloatTok{4.} \OtherTok{\textless{}{-}}\NormalTok{ df2}\SpecialCharTok{$}\NormalTok{A16.}\FloatTok{4.}\NormalTok{\_reversed}
\NormalTok{df2}\SpecialCharTok{$}\NormalTok{A17.}\FloatTok{10.} \OtherTok{\textless{}{-}}\NormalTok{ df2}\SpecialCharTok{$}\NormalTok{A17.}\FloatTok{10.}\NormalTok{\_reversed}


\CommentTok{\# Delete the reversed score columns at the end so it is less confusing and assign this to new dataframe. }

\NormalTok{df3 }\OtherTok{\textless{}{-}} \FunctionTok{select}\NormalTok{(df2, }\SpecialCharTok{{-}}\FunctionTok{c}\NormalTok{(A5.}\FloatTok{3.}\NormalTok{\_reversed, A16.}\FloatTok{3.}\NormalTok{\_reversed, A16.}\FloatTok{4.}\NormalTok{\_reversed, A17.}\FloatTok{10.}\NormalTok{\_reversed))}

\CommentTok{\# Check that this has transferred correctly. }

\FunctionTok{head}\NormalTok{(df3)}
\end{Highlighting}
\end{Shaded}

\begin{verbatim}
##   RespondentID Country A5.1 A5.2 A5.3. A5.4 A5.5 A16.1 A16.2 A16.3. A16.4.
## 1            1       1    7    3    10    3    8     5     2      2      2
## 2            2       1   10   10    10    8    1     4     8      9      9
## 3            3       1   10   10    10   10    7     7     7      4      8
## 4            4       1    8    8     2    7    7     7     8      9      9
## 5            5       1    7    7     9    6    3     6     2      7      7
## 6            6       1    8   10     2    9    9     9    10      7      4
##   A16.5 A16.6 A17.1 A17.2 A17.3 A17.4 A17.5 A17.6 A17.7 A17.8 A17.9 A17.10.
## 1     7     7     5     2     2     5     3     2     6    10     8       2
## 2    10     9     9     9     8     9     6     5     1     1     1       9
## 3    10    10     8    10     6     8    10    10    10     7    10       7
## 4     8     8     8     8     9     8     7     9     7     8     7       9
## 5     7     5     4     3     8     6     4     3     7     2     2       8
## 6     5     7     9     6     9    10     8    10     8    10     9       7
##   A17.11 A17.12
## 1      7      7
## 2      9      8
## 3     10      9
## 4      8      8
## 5      5      3
## 6      6      9
\end{verbatim}

\begin{Shaded}
\begin{Highlighting}[]
\CommentTok{\# Calculate mean score for anxiety around Covid{-}19 and the mean for threat variables for each participant.}

\NormalTok{anxiety\_mean }\OtherTok{\textless{}{-}} \FunctionTok{rowMeans}\NormalTok{(df3[, }\FunctionTok{c}\NormalTok{(}\StringTok{"A5.1"}\NormalTok{,}\StringTok{"A5.2"}\NormalTok{, }\StringTok{"A5.3."}\NormalTok{, }\StringTok{"A5.4"}\NormalTok{, }\StringTok{"A5.5"}\NormalTok{)]) }\CommentTok{\#, na.rm = TRUE)}

\NormalTok{c19\_threat\_mean }\OtherTok{\textless{}{-}} \FunctionTok{rowMeans}\NormalTok{(df3[, }\FunctionTok{c}\NormalTok{(}\StringTok{"A16.1"}\NormalTok{, }\StringTok{"A16.2"}\NormalTok{, }\StringTok{"A16.3."}\NormalTok{, }\StringTok{"A16.4."}\NormalTok{, }\StringTok{"A16.5"}\NormalTok{, }\StringTok{"A16.6"}\NormalTok{)]) }\CommentTok{\#, na.rm = TRUE)}

\NormalTok{safety\_threat\_mean }\OtherTok{\textless{}{-}}  \FunctionTok{rowMeans}\NormalTok{(df3[, }\FunctionTok{c}\NormalTok{(}\StringTok{"A17.1"}\NormalTok{, }\StringTok{"A17.2"}\NormalTok{, }\StringTok{"A17.3"}\NormalTok{, }\StringTok{"A17.4"}\NormalTok{, }\StringTok{"A17.5"}\NormalTok{, }\StringTok{"A17.6"}\NormalTok{, }\StringTok{"A17.7"}\NormalTok{, }\StringTok{"A17.8"}\NormalTok{, }\StringTok{"A17.9"}\NormalTok{, }\StringTok{"A17.10."}\NormalTok{, }\StringTok{"A17.11"}\NormalTok{, }\StringTok{"A17.12"}\NormalTok{)]) }\CommentTok{\#, na.rm = TRUE)}


\CommentTok{\# Add mean score for anxiety to dataframe}

\NormalTok{df3}\SpecialCharTok{$}\NormalTok{anxiety\_mean }\OtherTok{\textless{}{-}}\NormalTok{ anxiety\_mean}
\NormalTok{df3}\SpecialCharTok{$}\NormalTok{c19\_threat\_mean }\OtherTok{\textless{}{-}}\NormalTok{ c19\_threat\_mean}
\NormalTok{df3}\SpecialCharTok{$}\NormalTok{safety\_threat\_mean }\OtherTok{\textless{}{-}}\NormalTok{ safety\_threat\_mean}


\CommentTok{\# Remove the individual questionnaire scores now that the mean has been calculated for the variables of interest.}

\NormalTok{df4 }\OtherTok{\textless{}{-}}\NormalTok{ df3[, }\SpecialCharTok{{-}}\FunctionTok{c}\NormalTok{(}\DecValTok{3}\SpecialCharTok{:}\DecValTok{25}\NormalTok{)]}

\CommentTok{\# Recode the country column}

\NormalTok{df5 }\OtherTok{\textless{}{-}}\NormalTok{ df4 }\SpecialCharTok{\%\textgreater{}\%} 
  \FunctionTok{mutate}\NormalTok{(}\AttributeTok{Country =} \FunctionTok{recode}\NormalTok{(Country,}
                          \StringTok{\textquotesingle{}1\textquotesingle{}} \OtherTok{=} \StringTok{"Germany"}\NormalTok{,}
                          \StringTok{\textquotesingle{}2\textquotesingle{}} \OtherTok{=} \StringTok{"Spain"}\NormalTok{,}
                          \StringTok{\textquotesingle{}3\textquotesingle{}} \OtherTok{=} \StringTok{"Netherlands"}\NormalTok{,}
                          \StringTok{\textquotesingle{}4\textquotesingle{}} \OtherTok{=} \StringTok{"UK"}\NormalTok{))}

\CommentTok{\# Check that the new dataframe has all the relevant information needed. }


\FunctionTok{head}\NormalTok{(df5)}
\end{Highlighting}
\end{Shaded}

\begin{verbatim}
##   RespondentID Country anxiety_mean c19_threat_mean safety_threat_mean
## 1            1 Germany          6.2        4.166667           4.916667
## 2            2 Germany          7.8        8.166667           6.250000
## 3            3 Germany          9.4        7.666667           8.750000
## 4            4 Germany          6.4        8.166667           8.000000
## 5            5 Germany          6.4        5.666667           4.583333
## 6            6 Germany          7.6        7.000000           8.416667
\end{verbatim}

\begin{Shaded}
\begin{Highlighting}[]
\CommentTok{\#As df5 is in wide format, it will need to be converted to long format. }

\NormalTok{long\_data }\OtherTok{\textless{}{-}}\NormalTok{ df5}\SpecialCharTok{\%\textgreater{}\%}
  \FunctionTok{pivot\_longer}\NormalTok{(}\AttributeTok{cols =} \FunctionTok{c}\NormalTok{(}\StringTok{"anxiety\_mean"}\NormalTok{, }\StringTok{"c19\_threat\_mean"}\NormalTok{, }\StringTok{"safety\_threat\_mean"}\NormalTok{), }
               \AttributeTok{names\_to =} \StringTok{"Variable"}\NormalTok{,}
               \AttributeTok{values\_to =} \StringTok{"Score"}\NormalTok{)}
\FunctionTok{head}\NormalTok{(long\_data)}
\end{Highlighting}
\end{Shaded}

\begin{verbatim}
## # A tibble: 6 x 4
##   RespondentID Country Variable           Score
##          <int> <chr>   <chr>              <dbl>
## 1            1 Germany anxiety_mean        6.2 
## 2            1 Germany c19_threat_mean     4.17
## 3            1 Germany safety_threat_mean  4.92
## 4            2 Germany anxiety_mean        7.8 
## 5            2 Germany c19_threat_mean     8.17
## 6            2 Germany safety_threat_mean  6.25
\end{verbatim}

\begin{Shaded}
\begin{Highlighting}[]
\CommentTok{\# Aggregate variables to produce a summaried overall mean score for each variable according to the country. }

\NormalTok{data\_summary }\OtherTok{\textless{}{-}}\NormalTok{  long\_data }\SpecialCharTok{\%\textgreater{}\%}
  \FunctionTok{group\_by}\NormalTok{(Country, Variable) }\SpecialCharTok{\%\textgreater{}\%} 
  \FunctionTok{summarise}\NormalTok{( }\AttributeTok{mean\_score=} \FunctionTok{mean}\NormalTok{(Score), }\AttributeTok{.groups =} \StringTok{"drop"}\NormalTok{)}


\CommentTok{\# Creating factors as the countries are categorical data and to prevent issues when creating the plot later on. }

\NormalTok{data\_summary }\OtherTok{\textless{}{-}}\NormalTok{ data\_summary }\SpecialCharTok{\%\textgreater{}\%} 
  \FunctionTok{mutate}\NormalTok{(}\AttributeTok{Country =} \FunctionTok{factor}\NormalTok{ (Country, }\AttributeTok{levels =} \FunctionTok{c}\NormalTok{(}\StringTok{"Germany"}\NormalTok{, }\StringTok{"Netherlands"}\NormalTok{, }\StringTok{"Spain"}\NormalTok{, }\StringTok{"UK"}\NormalTok{)))}

\NormalTok{data\_summary}
\end{Highlighting}
\end{Shaded}

\begin{verbatim}
## # A tibble: 12 x 3
##    Country     Variable           mean_score
##    <fct>       <chr>                   <dbl>
##  1 Germany     anxiety_mean             5.88
##  2 Germany     c19_threat_mean          5.90
##  3 Germany     safety_threat_mean       5.39
##  4 Netherlands anxiety_mean             5.87
##  5 Netherlands c19_threat_mean          6.21
##  6 Netherlands safety_threat_mean       5.47
##  7 Spain       anxiety_mean             6.58
##  8 Spain       c19_threat_mean          6.47
##  9 Spain       safety_threat_mean       5.35
## 10 UK          anxiety_mean             6.48
## 11 UK          c19_threat_mean          6.37
## 12 UK          safety_threat_mean       5.66
\end{verbatim}

\subsection{Visualisation}\label{visualisation}

The best way to visualise this data may be through a bar chart with the
countries as the categorical variable and the questionnaire scores as
continuous. With this, it is possible to check if the predictions made
about the data,as stated previosly, are met.

\begin{Shaded}
\begin{Highlighting}[]
\CommentTok{\# Create a bar chart of anxiety and threat estimation scores for each country.}

\NormalTok{p1 }\OtherTok{\textless{}{-}} \FunctionTok{ggplot}\NormalTok{(data\_summary, }\FunctionTok{aes}\NormalTok{(}\AttributeTok{x =}\NormalTok{ Country, }\AttributeTok{y =}\NormalTok{ mean\_score, }\AttributeTok{fill =}\NormalTok{ Variable)) }\SpecialCharTok{+}
  \FunctionTok{geom\_bar}\NormalTok{(}\AttributeTok{stat =} \StringTok{"identity"}\NormalTok{, }\AttributeTok{position =} \StringTok{"dodge"}\NormalTok{, }\AttributeTok{colour =} \StringTok{"black"}\NormalTok{) }\SpecialCharTok{+}
  
  \CommentTok{\# Set Y{-}axis scale from 0 {-}10. }
  
  \FunctionTok{ylim}\NormalTok{(}\DecValTok{0}\NormalTok{, }\DecValTok{10}\NormalTok{) }\SpecialCharTok{+}
  
  \CommentTok{\# Add labels}
  
  \FunctionTok{labs}\NormalTok{(}
    \AttributeTok{title =} \StringTok{"Visualisation of Social and Psychological Reactions During the Covid{-}19}\SpecialCharTok{\textbackslash{}n}\StringTok{ Pandemic Across Four European Countries"}\NormalTok{,}
    \AttributeTok{subtitle =} \StringTok{"Comparing anxiety levels and threat estimation in relation to Covid{-}19 and general threat"}\NormalTok{,}
    \AttributeTok{x =} \StringTok{"Country"}\NormalTok{,}
    \AttributeTok{y =} \StringTok{"Mean score on questionnaires"}\NormalTok{,}
    \AttributeTok{fill =} \StringTok{"Questionnaire Items"}\NormalTok{,}
    \AttributeTok{caption =} \StringTok{"Source: A Dataset of Social{-}Psychological and Emotional Reactions During the COVID{-}19 Pandemic Across Four European Countries (Abadi et al., 2023)"}
\NormalTok{  ) }\SpecialCharTok{+}
  
  \CommentTok{\# Customise by adding complementary colours to aid better viewing experience. }
  
  \FunctionTok{scale\_fill\_manual}\NormalTok{(}
    \AttributeTok{values =} \FunctionTok{c}\NormalTok{(}\StringTok{"anxiety\_mean"} \OtherTok{=} \StringTok{"\#c6cf95"}\NormalTok{, }\StringTok{"c19\_threat\_mean"} \OtherTok{=} \StringTok{"\#7fcdbb"}\NormalTok{, }\StringTok{"safety\_threat\_mean"} \OtherTok{=} \StringTok{"\#2c7fb8"}\NormalTok{),}
    \AttributeTok{labels =} \FunctionTok{c}\NormalTok{(}\StringTok{"Anxiety rating"}\NormalTok{, }\StringTok{"COVID{-}19 threat estimation"}\NormalTok{, }\StringTok{"General threat estimation"}\NormalTok{)}
\NormalTok{  ) }\SpecialCharTok{+}
  
  
  \FunctionTok{theme}\NormalTok{(}
    \AttributeTok{plot.title =} \FunctionTok{element\_text}\NormalTok{(}\AttributeTok{hjust =} \FloatTok{0.5}\NormalTok{, }\AttributeTok{size =} \DecValTok{14}\NormalTok{, }\AttributeTok{face =} \StringTok{"bold"}\NormalTok{),}
    \AttributeTok{plot.subtitle =} \FunctionTok{element\_text}\NormalTok{(}\AttributeTok{hjust =} \FloatTok{0.5}\NormalTok{, }\AttributeTok{size =} \DecValTok{12}\NormalTok{),}
    \AttributeTok{plot.caption =} \FunctionTok{element\_text}\NormalTok{(}\AttributeTok{hjust =} \DecValTok{0}\NormalTok{, }\AttributeTok{face =} \StringTok{"italic"}\NormalTok{),}
    \AttributeTok{axis.title =} \FunctionTok{element\_text}\NormalTok{(}\AttributeTok{size =} \DecValTok{12}\NormalTok{),}
    \AttributeTok{legend.title =} \FunctionTok{element\_text}\NormalTok{(}\AttributeTok{size =} \DecValTok{12}\NormalTok{),}
    \AttributeTok{legend.text =} \FunctionTok{element\_text}\NormalTok{(}\AttributeTok{size =} \DecValTok{10}\NormalTok{),}
    \AttributeTok{panel.background =} \FunctionTok{element\_rect}\NormalTok{(}\AttributeTok{fill =} \StringTok{"\#ffffff"}\NormalTok{, }\AttributeTok{color =} \ConstantTok{NA}\NormalTok{),}
    
    \CommentTok{\# Remove the vertical grid lines}
    \AttributeTok{panel.grid.major.x =} \FunctionTok{element\_blank}\NormalTok{(),}
    \AttributeTok{panel.grid.minor.x =} \FunctionTok{element\_blank}\NormalTok{(),}
    
    \CommentTok{\# Keep the horizontal grid lines}
    \AttributeTok{panel.grid.major.y =} \FunctionTok{element\_line}\NormalTok{(}\AttributeTok{color =} \StringTok{"grey"}\NormalTok{, }\AttributeTok{linewidth =} \FloatTok{0.5}\NormalTok{),}
    
    \CommentTok{\# Adding a secondary line makes it easier to estimate the values on the y{-}axis. }
    \AttributeTok{panel.grid.minor.y =} \FunctionTok{element\_line}\NormalTok{(}\AttributeTok{color =} \StringTok{"lightgrey"}\NormalTok{, }\AttributeTok{linewidth =} \FloatTok{0.25}\NormalTok{),}
   
     \CommentTok{\# Adjust margins}
    \AttributeTok{plot.margin =} \FunctionTok{margin}\NormalTok{(}\DecValTok{10}\NormalTok{, }\DecValTok{10}\NormalTok{, }\DecValTok{10}\NormalTok{, }\DecValTok{10}\NormalTok{))}
\NormalTok{p1}
\end{Highlighting}
\end{Shaded}

\includegraphics{index_files/figure-latex/visualisation 1-1.pdf}

\begin{Shaded}
\begin{Highlighting}[]
\CommentTok{\# plot that with the weekly count cases of c{-}19 for each country. }
\CommentTok{\# the mean anxiety and threat score for each country as a whole needs to be calculated. }
\end{Highlighting}
\end{Shaded}

\begin{Shaded}
\begin{Highlighting}[]
\CommentTok{\#ggsave(here("visualisations/p1.png"))}
\end{Highlighting}
\end{Shaded}

As predicted, in looking at the average anxiety scores, Spain had
marginally higher rating followed by the UK then Germany and
Netherlands. In terms of average Covid-19 threat detection score, Spain
had the highest scores, followed by the UK, Netherlands and Germany. In
terms of general threat estimation, UK scored the highest and Spain the
lowest.

\subsubsection{Import data (part 2)}\label{import-data-part-2}

It will be helpful to see the relationship between these findings to
that of the total number of Covid- 19 related deaths for each country.
According to Abadi et al., (2023) the data collection for their study
took place in April 2020. The European Centre for Disease Prevention and
Control (ECDC) provides data on the weekly number of Covid-19 related
cases and deaths worldwide.The data for April 2020 is also reported and
is available to download from their official website:
\url{https://www.ecdc.europa.eu/en/publications-data/download-historical-data-20-june-2022-weekly-number-new-reported-covid-19-cases}

\begin{Shaded}
\begin{Highlighting}[]
\NormalTok{ecdc\_raw }\OtherTok{\textless{}{-}} \FunctionTok{read.csv}\NormalTok{(}\StringTok{"https://opendata.ecdc.europa.eu/covid19/casedistribution/csv"}\NormalTok{, }\AttributeTok{na.strings =} \StringTok{""}\NormalTok{, }\AttributeTok{fileEncoding =} \StringTok{"UTF{-}8{-}BOM"}\NormalTok{)}

\FunctionTok{head}\NormalTok{(ecdc\_raw)}
\end{Highlighting}
\end{Shaded}

\begin{verbatim}
##      dateRep day month year cases deaths countriesAndTerritories geoId
## 1 14/12/2020  14    12 2020   746      6             Afghanistan    AF
## 2 13/12/2020  13    12 2020   298      9             Afghanistan    AF
## 3 12/12/2020  12    12 2020   113     11             Afghanistan    AF
## 4 11/12/2020  11    12 2020    63     10             Afghanistan    AF
## 5 10/12/2020  10    12 2020   202     16             Afghanistan    AF
## 6 09/12/2020   9    12 2020   135     13             Afghanistan    AF
##   countryterritoryCode popData2019 continentExp
## 1                  AFG    38041757         Asia
## 2                  AFG    38041757         Asia
## 3                  AFG    38041757         Asia
## 4                  AFG    38041757         Asia
## 5                  AFG    38041757         Asia
## 6                  AFG    38041757         Asia
##   Cumulative_number_for_14_days_of_COVID.19_cases_per_100000
## 1                                                   9.013779
## 2                                                   7.052776
## 3                                                   6.868768
## 4                                                   7.134266
## 5                                                   6.968658
## 6                                                   6.963401
\end{verbatim}

\subsubsection{Wrangle the dataset}\label{wrangle-the-dataset}

This dataset has information on all the countries in the world. For the
purpose of this visualisation, only the data for Germany, Netherlands,
Spain and UK are needed.

\begin{Shaded}
\begin{Highlighting}[]
\CommentTok{\# Remove all the other countries}

\NormalTok{ecdc2 }\OtherTok{\textless{}{-}}\NormalTok{ ecdc\_raw }\SpecialCharTok{\%\textgreater{}\%} 
  \FunctionTok{filter}\NormalTok{(countriesAndTerritories }\SpecialCharTok{\%in\%} \FunctionTok{c}\NormalTok{(}\StringTok{"Germany"}\NormalTok{, }\StringTok{"Netherlands"}\NormalTok{, }\StringTok{"Spain"}\NormalTok{, }\StringTok{"United\_Kingdom"}\NormalTok{))}

\CommentTok{\# Remove data for all other months, except April. }

\NormalTok{ecdc2 }\OtherTok{\textless{}{-}}\NormalTok{ ecdc2 }\SpecialCharTok{\%\textgreater{}\%} 
  \FunctionTok{filter}\NormalTok{(month }\SpecialCharTok{==} \DecValTok{4}\NormalTok{)}

\CommentTok{\# Remove the other wanted variables also. }

\NormalTok{ecdc3 }\OtherTok{\textless{}{-}}\NormalTok{ ecdc2 }\SpecialCharTok{\%\textgreater{}\%} 
  \FunctionTok{select}\NormalTok{(}\SpecialCharTok{{-}}\NormalTok{day, }\SpecialCharTok{{-}}\NormalTok{month, }\SpecialCharTok{{-}}\NormalTok{year, }\SpecialCharTok{{-}}\NormalTok{cases, }\SpecialCharTok{{-}}\NormalTok{geoId, }\SpecialCharTok{{-}}\NormalTok{countryterritoryCode, }\SpecialCharTok{{-}}\NormalTok{popData2019, }\SpecialCharTok{{-}}\NormalTok{continentExp)}

\CommentTok{\# Calucate the total deaths for April for each country.}

\NormalTok{total\_deaths }\OtherTok{\textless{}{-}}\NormalTok{ ecdc3 }\SpecialCharTok{\%\textgreater{}\%} 
  \FunctionTok{group\_by}\NormalTok{(countriesAndTerritories) }\SpecialCharTok{\%\textgreater{}\%} 
  \FunctionTok{summarise}\NormalTok{(}\AttributeTok{total\_deaths =} \FunctionTok{sum}\NormalTok{(deaths, }\AttributeTok{na.rm =} \ConstantTok{TRUE}\NormalTok{))}

\CommentTok{\# Add total deaths column to dataframe.}

\NormalTok{ecdc3 }\OtherTok{\textless{}{-}}\NormalTok{ ecdc3 }\SpecialCharTok{\%\textgreater{}\%} 
  \FunctionTok{left\_join}\NormalTok{(total\_deaths, }\AttributeTok{by =} \StringTok{"countriesAndTerritories"}\NormalTok{)}


\CommentTok{\# Table to display the total deaths calculated for each country in April 2020 when the survey was conducted. }

\NormalTok{ecdc\_summary }\OtherTok{\textless{}{-}}\NormalTok{ ecdc3 }\SpecialCharTok{\%\textgreater{}\%}
  \FunctionTok{group\_by}\NormalTok{(countriesAndTerritories, total\_deaths) }\SpecialCharTok{\%\textgreater{}\%}
  \FunctionTok{summarise}\NormalTok{(}\AttributeTok{count =} \FunctionTok{n}\NormalTok{(), }\AttributeTok{.groups =} \StringTok{"drop"}\NormalTok{)}\SpecialCharTok{\%\textgreater{}\%}
  \FunctionTok{select}\NormalTok{(}\SpecialCharTok{{-}}\NormalTok{count)}

\NormalTok{ecdc\_summary}
\end{Highlighting}
\end{Shaded}

\begin{verbatim}
## # A tibble: 4 x 2
##   countriesAndTerritories total_deaths
##   <chr>                          <int>
## 1 Germany                         5705
## 2 Netherlands                     3847
## 3 Spain                          17203
## 4 United_Kingdom                 23999
\end{verbatim}

\begin{Shaded}
\begin{Highlighting}[]
\CommentTok{\# Ensure that the names match for the countires}

\NormalTok{ecdc\_summary }\OtherTok{\textless{}{-}}\NormalTok{ ecdc\_summary }\SpecialCharTok{\%\textgreater{}\%}
  \FunctionTok{mutate}\NormalTok{(}\AttributeTok{countriesAndTerritories =} \FunctionTok{if\_else}\NormalTok{(countriesAndTerritories }\SpecialCharTok{==} \StringTok{"United\_Kingdom"}\NormalTok{, }\StringTok{"UK"}\NormalTok{, countriesAndTerritories))}

\NormalTok{ecdc\_summary}
\end{Highlighting}
\end{Shaded}

\begin{verbatim}
## # A tibble: 4 x 2
##   countriesAndTerritories total_deaths
##   <chr>                          <int>
## 1 Germany                         5705
## 2 Netherlands                     3847
## 3 Spain                          17203
## 4 UK                             23999
\end{verbatim}

\subsection{Final Visualisation}\label{final-visualisation}

The final visualisation will comprise of the previous graph overlaid
with the data around the of number of deaths for each of the four
countries.

\begin{Shaded}
\begin{Highlighting}[]
  \CommentTok{\# Adding data around total number of deaths reported to the previous graph. }

\NormalTok{p2 }\OtherTok{\textless{}{-}} \FunctionTok{ggplot}\NormalTok{(data\_summary, }\FunctionTok{aes}\NormalTok{(}\AttributeTok{x =}\NormalTok{ Country, }\AttributeTok{y =}\NormalTok{ mean\_score, }\AttributeTok{fill =}\NormalTok{ Variable)) }\SpecialCharTok{+}
  \CommentTok{\# Bar chart layer}
  \FunctionTok{geom\_bar}\NormalTok{(}\AttributeTok{stat =} \StringTok{"identity"}\NormalTok{, }\AttributeTok{position =} \StringTok{"dodge"}\NormalTok{, }\AttributeTok{colour =} \StringTok{"black"}\NormalTok{) }\SpecialCharTok{+}
  
  \CommentTok{\# Scatter plot for EDCD data and define the scale for the y{-}axis. }
  \FunctionTok{geom\_point}\NormalTok{(}
    \AttributeTok{data =}\NormalTok{ ecdc\_summary,  }
    \FunctionTok{aes}\NormalTok{(}\AttributeTok{x =}\NormalTok{ countriesAndTerritories, }\AttributeTok{y =}\NormalTok{ (total\_deaths }\SpecialCharTok{{-}} \DecValTok{3500}\NormalTok{) }\SpecialCharTok{*}\NormalTok{ (}\DecValTok{10} \SpecialCharTok{/} \DecValTok{21500}\NormalTok{), }\AttributeTok{color =} \StringTok{"Total Deaths"}\NormalTok{, }\AttributeTok{text =} \FunctionTok{paste}\NormalTok{(}\StringTok{"}\SpecialCharTok{\textbackslash{}n}\StringTok{Country: "}\NormalTok{, countriesAndTerritories, }\StringTok{"}\SpecialCharTok{\textbackslash{}n}\StringTok{Total Deaths: "}\NormalTok{, total\_deaths)),  }
    \AttributeTok{size =} \DecValTok{3}\NormalTok{,}
    \AttributeTok{inherit.aes =} \ConstantTok{FALSE}\NormalTok{,}
    \AttributeTok{show.legend =} \ConstantTok{FALSE}\NormalTok{) }\SpecialCharTok{+}
  
  \CommentTok{\# Add dashed red line through the points}
  \FunctionTok{geom\_line}\NormalTok{(}
    \AttributeTok{data =}\NormalTok{ ecdc\_summary,}
    \FunctionTok{aes}\NormalTok{(}\AttributeTok{x =} \FunctionTok{as.numeric}\NormalTok{(}\FunctionTok{factor}\NormalTok{(countriesAndTerritories)), }\AttributeTok{y =}\NormalTok{ (total\_deaths }\SpecialCharTok{{-}} \DecValTok{3500}\NormalTok{) }\SpecialCharTok{*}\NormalTok{ (}\DecValTok{10} \SpecialCharTok{/} \DecValTok{21500}\NormalTok{)),}
    \AttributeTok{color =} \StringTok{"red"}\NormalTok{,}
    \AttributeTok{linetype =} \StringTok{"dashed"}\NormalTok{,}
    
  \CommentTok{\# Remove the legend as the axis for the secondary axis will be red}
    \AttributeTok{inherit.aes =} \ConstantTok{FALSE}\NormalTok{,}
    \AttributeTok{show.legend =} \ConstantTok{FALSE}\NormalTok{) }\SpecialCharTok{+}
  
  \CommentTok{\# Specify the primary and secondary y{-}axis}
    \FunctionTok{scale\_y\_continuous}\NormalTok{(}
    \AttributeTok{name =} \StringTok{"Mean score on questionnaires"}\NormalTok{,}
    \AttributeTok{limits =} \FunctionTok{c}\NormalTok{(}\DecValTok{0}\NormalTok{, }\DecValTok{10}\NormalTok{),}
    
  \CommentTok{\# Add second axis and specify the scale. }
    \AttributeTok{sec.axis =} \FunctionTok{sec\_axis}\NormalTok{(}\SpecialCharTok{\textasciitilde{}}\NormalTok{ . }\SpecialCharTok{*} \DecValTok{21500} \SpecialCharTok{/} \DecValTok{10} \SpecialCharTok{+} \DecValTok{3500}\NormalTok{, }
      \AttributeTok{name =} \StringTok{"Total number of C{-}19 deaths reported"}\NormalTok{,}
      \AttributeTok{labels =}\NormalTok{ scales}\SpecialCharTok{::}\FunctionTok{label\_number}\NormalTok{()))}\SpecialCharTok{+}
  
  \CommentTok{\# Labels and colors}
  \FunctionTok{labs}\NormalTok{(}
    \AttributeTok{title =} \StringTok{"Visualisation of Social and Psychological Reactions During the Covid{-}19}\SpecialCharTok{\textbackslash{}n}\StringTok{ Pandemic Across Four European Countries"}\NormalTok{,}
    \AttributeTok{subtitle =} \StringTok{"Comparing anxiety levels and threat estimation in relation to Covid{-}19 and general threat"}\NormalTok{,}
    \AttributeTok{x =} \StringTok{"Country"}\NormalTok{,}
    \AttributeTok{fill =} \StringTok{"Questionnaire Items:  "}\NormalTok{,}
    \AttributeTok{caption =} \StringTok{"Source: A Dataset of Social{-}Psychological and Emotional Reactions During the COVID{-}19 Pandemic Across Four European Countries (Abadi et al., 2023).}\SpecialCharTok{\textbackslash{}n}\StringTok{Data on total number of deaths as reported by the ECDC for the month of April 2020."}
\NormalTok{  ) }\SpecialCharTok{+}
  \FunctionTok{scale\_fill\_manual}\NormalTok{(}
    \AttributeTok{values =} \FunctionTok{c}\NormalTok{(}\StringTok{"anxiety\_mean"} \OtherTok{=} \StringTok{"\#c6cf95"}\NormalTok{, }\StringTok{"c19\_threat\_mean"} \OtherTok{=} \StringTok{"\#7fcdbb"}\NormalTok{, }\StringTok{"safety\_threat\_mean"} \OtherTok{=} \StringTok{"\#2c7fb8"}\NormalTok{),}
    \AttributeTok{labels =} \FunctionTok{c}\NormalTok{(}\StringTok{"Anxiety rating"}\NormalTok{, }\StringTok{"COVID{-}19 threat estimation"}\NormalTok{, }\StringTok{"General threat estimation"}\NormalTok{)}
\NormalTok{  ) }\SpecialCharTok{+}
  
  \CommentTok{\# Customise the themes}
  
  \FunctionTok{theme}\NormalTok{(}
    \AttributeTok{plot.title =} \FunctionTok{element\_text}\NormalTok{(}\AttributeTok{hjust =} \FloatTok{0.5}\NormalTok{, }\AttributeTok{size =} \DecValTok{14}\NormalTok{, }\AttributeTok{face =} \StringTok{"bold"}\NormalTok{),}
    \AttributeTok{plot.subtitle =} \FunctionTok{element\_text}\NormalTok{(}\AttributeTok{hjust =} \FloatTok{0.5}\NormalTok{, }\AttributeTok{size =} \DecValTok{12}\NormalTok{),}
    \AttributeTok{plot.caption =} \FunctionTok{element\_text}\NormalTok{(}\AttributeTok{hjust =} \DecValTok{0}\NormalTok{, }\AttributeTok{face =} \StringTok{"italic"}\NormalTok{), }
    
    
  \CommentTok{\# Make the secondary axis and text red so it corresponds with the dashed red line for the death numbers. }
    \AttributeTok{axis.title.y.right =} \FunctionTok{element\_text}\NormalTok{(}\AttributeTok{size =} \DecValTok{12}\NormalTok{, }\AttributeTok{color =} \StringTok{"red"}\NormalTok{),  }
    \AttributeTok{axis.text.y.right =} \FunctionTok{element\_text}\NormalTok{(}\AttributeTok{color =} \StringTok{"red"}\NormalTok{),}
    \AttributeTok{axis.title =} \FunctionTok{element\_text}\NormalTok{(}\AttributeTok{size =} \DecValTok{12}\NormalTok{),}
    \AttributeTok{axis.ticks.x =} \FunctionTok{element\_blank}\NormalTok{(),}
    
  \CommentTok{\# Adjust the legend}
    \AttributeTok{legend.title =} \FunctionTok{element\_text}\NormalTok{(}\AttributeTok{size =} \DecValTok{10}\NormalTok{),}
    \AttributeTok{legend.text =} \FunctionTok{element\_text}\NormalTok{(}\AttributeTok{size =} \DecValTok{8}\NormalTok{),}
    \AttributeTok{legend.position =} \FunctionTok{c}\NormalTok{(}\FloatTok{0.15}\NormalTok{, }\FloatTok{0.82}\NormalTok{),}
    \AttributeTok{legend.key =} \FunctionTok{element\_rect}\NormalTok{(}\AttributeTok{colour =} \ConstantTok{NA}\NormalTok{, }\AttributeTok{fill =} \ConstantTok{NA}\NormalTok{),}
    
    
 \CommentTok{\# Add white background}
    \AttributeTok{panel.background =} \FunctionTok{element\_rect}\NormalTok{(}\AttributeTok{fill =} \StringTok{"\#ffffff"}\NormalTok{, }\AttributeTok{colour =} \ConstantTok{NA}\NormalTok{),}
    \AttributeTok{panel.grid.major.x =} \FunctionTok{element\_blank}\NormalTok{(),}
    \AttributeTok{panel.grid.minor.x =} \FunctionTok{element\_blank}\NormalTok{(),}
   
  \CommentTok{\# Keep horizontal lines}
    \AttributeTok{panel.grid.major.y =} \FunctionTok{element\_line}\NormalTok{(}\AttributeTok{color =} \StringTok{"grey"}\NormalTok{, }\AttributeTok{linewidth =} \FloatTok{0.5}\NormalTok{),}
    
  \CommentTok{\# Adding a secondary horizontal line makes it easier to estimate the values on the y{-}axis. }
    \AttributeTok{panel.grid.minor.y =} \FunctionTok{element\_line}\NormalTok{(}\AttributeTok{color =} \StringTok{"lightgrey"}\NormalTok{, }\AttributeTok{linewidth =} \FloatTok{0.25}\NormalTok{),}
    
  \CommentTok{\#Adjust margins}
    \AttributeTok{plot.margin =} \FunctionTok{margin}\NormalTok{(}\DecValTok{30}\NormalTok{, }\DecValTok{20}\NormalTok{, }\DecValTok{20}\NormalTok{, }\DecValTok{30}\NormalTok{))}
\end{Highlighting}
\end{Shaded}

\begin{verbatim}
## Warning in geom_point(data = ecdc_summary, aes(x = countriesAndTerritories, :
## Ignoring unknown aesthetics: text
\end{verbatim}

\begin{verbatim}
## Warning: A numeric `legend.position` argument in `theme()` was deprecated in ggplot2
## 3.5.0.
## i Please use the `legend.position.inside` argument of `theme()` instead.
## This warning is displayed once every 8 hours.
## Call `lifecycle::last_lifecycle_warnings()` to see where this warning was
## generated.
\end{verbatim}

\begin{Shaded}
\begin{Highlighting}[]
\NormalTok{p2}
\end{Highlighting}
\end{Shaded}

\includegraphics{index_files/figure-latex/visualisation 2-1.pdf}

\begin{Shaded}
\begin{Highlighting}[]
\CommentTok{\#ggsave(here("visualisations/p2.png"), width = 13, height = 8.5)}
\end{Highlighting}
\end{Shaded}

\subsection{Summary}\label{summary}

The results suggest that participants from Spain reported the highest
levels of anxiety and the Netherlands reported the lowest levels
according to the questionnaire results. Participants from Spain reported
high levels of threat estimation in terms of Covid-19.

An obvious difference that stood out from this visualisation is the
total number of Covid-19 related deaths for each of the four countries.
Abadi et al., (2023) reported that at the time of their survey, Spain
had the highest number of deaths but the ECDC data indicates that the UK
had the highest number of Covid-19 related deaths for the month of April
2020.

The authors report that they obtained a representative sample from each
of the countries (approximately 500 people from each country). From this
data, there is no significant differences between the countries for
these variables that can be observed. Future research may consider
recruiting a larger sample which may produced much clearer distinction
of the differences in scores. It may be that participant characteristics
may have impacted the outcomes on the survey, e.g., voluntary sample
that was willing to complete the surveys.

\subsection{Follow up}\label{follow-up}

This visualisation may be made better with the addition of an
interactive element. I had attempted to make the graph interactive via
ggplotly, however I found that this removed some of the helpful labels
that I felt was important to the visualisation. It also seemed to remove
the secondary axis. Although with the interactive plot, when the cursor
hovers over the point, the information is provided, it was felt that
this may be confusing at first glance without the secondary axis and may
lead to the graph being read wrong.

If additional data is acquired, for example, obtaining data from more
countries in Europe and how the population levels of anxiety changed
across may be good to visualise in terms of an interactive line or bar
chart. It will also be good to potentially visualise the information of
all the countries in Europe in a map.

\subsection{References}\label{references}

Abadi, D., Arnaldo, I., \& Fischer,A. (2023). A Dataset of
SocialPsychological and EmotionalReactions During the COVID-19 Pandemic
Across FourEuropean Countries. \emph{Journal of Open Psychology Data},
11: 11, pp.~1--11. DOI: \url{https://doi.org/10.5334/jopd.86}

Miyah, Y., Benjelloun, M., Lairini, S., \& Lahrichi, A. (2022). COVID-19
impact on public health, environment, human psychology, global
socioeconomy, and education. \emph{The Scientific World Journal},
2022(5578284) 1-8. \url{https://doi.org/10.1155/2022/5578284}.

\end{document}
